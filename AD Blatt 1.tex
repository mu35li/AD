\documentclass[12pt,a4paper]{article}
\usepackage[utf8]{inputenc}
\usepackage[german]{babel}
\usepackage[T1]{fontenc}
\usepackage{amsmath}
\usepackage{amsfonts}
\usepackage{amssymb}
\author{Uschi Dolfus, Frederik Wille, Julian Deinert}
\title{Algorithmen und Datenstrukturen Aufgabenblatt 1}
\begin{document}
\maketitle

\begin{enumerate}
  \item[Aufgabe] 1
  \begin{enumerate}
    \item $\frac{1}{x} \prec 1 \prec \log(\log(x)) \prec \log(x) \asymp \log(x^{3}) \prec \log(x^{\log(x)}) \prec x^{0.01} \prec x^{\frac{1}{2}} \prec x(\log(x)) \prec x^{8} \prec 2^{x} \prec 8^{x} \prec x! \prec x^{x}  $ \\\\
	Beweise: \\
	\begin{enumerate}	    
    \item $\lim\limits_{x \to \infty}{\frac{\frac{1}{x}}{1}} = \lim\limits_{x \to \infty}{\frac{1}{x}} = 0$
	\item $\lim\limits_{x \to \infty}{\frac{1}{\log(\log(x))}} = 0$ 
	\item $\lim\limits_{x \to \infty}{\frac{\log(\log(x))}{\log(x)}} = 0 $, da $\log(x) < x \to \log(\log(x)) < \log(x)$
	\item $\lim\limits_{x \to \infty}{\frac{\log(x)}{\log(x^{3})}} = \lim\limits_{x \to \infty}{\frac{\log(x)}{3\log(x)}} = \frac{1}{3}$
	\item $\lim\limits_{x \to \infty}{\frac{\log(x^{3})}{\log(x^{\log(x)})}} = \lim\limits_{x \to \infty}{\frac{3\log(x)}{\log(x)^{2}}} = \lim\limits_{x \to \infty}{\frac{3}{\log(x)}} = 0 $
	\item $\lim\limits_{x \to \infty}{\frac{\log(x^{\log(x)})}{x^{0.01}}} = 0$, da jede Potenzfunktion schneller wächst, als jede Potenz vom Logarithmus
	\item $\lim\limits_{x \to \infty}{\frac{x^{0.01}}{x^{\frac{1}{2}}}} = 0$
	\item $\lim\limits_{x \to \infty}{\frac{x^{\frac{1}{2}}}{x\log(x)}} = \lim\limits_{x \to \infty}{\frac{x^{\frac{1}{2}}}{x^{\frac{1}{2}}\times x^{\frac{1}{2}}\times \log(x)}} = \lim\limits_{x \to \infty}{\frac{1}{x^{\frac{1}{2}}\log(x)}} = 0$
	\item $\lim\limits_{x \to \infty}{\frac{x\log(x)}{x^{8}}} = \frac{\log(x)}{x^{7}} = 0 $
	\item $\lim\limits_{x \to \infty}{\frac{x^{8}}{2^{x}}} = 0$
	\item $\lim\limits_{x \to \infty}{\frac{2^{x}}{8^{x}}} = \lim\limits_{x \to \infty}{(\frac{2}{8})^{x}} = \lim\limits_{x \to \infty}{(\frac{1}{4})^{x}} = \lim\limits_{x \to \infty}{\frac{1^{x}}{4^{x}}} = \lim\limits_{x \to \infty}{\frac{1}{4^{x}}} = 0$
	\item $\lim\limits_{x \to \infty}{\frac{8^{x}}{x!}} = \lim\limits_{x \to \infty}{\frac{8 \times 8\times 8 \dots \times 8}{x\times(x-1)\times(x-2)\times \dots \times1}} = 0 $
	\item $\lim\limits_{x \to \infty}{\frac{x!}{x^{x}}} = \frac{x\times(x-1)\times(x-2)\times \dots \times 1}{x\times x\times x\times \dots \times x} = 0$ 
    \end{enumerate}
    \item
    \begin{enumerate}
    \item 
    Behauptung: \\
    Sei $b$ beliebig und $b > 1$, so gilt $\log_{b}(n) \in \Theta(\log_{2}(n))$\\\\
Beweis:\\ Die Behauptung ist äquivalent zu: \\ $\lim\limits_{n \to \infty}{\frac{\log_{b}(n)}{\log_{2}(n)}} = $ konstanter Wert.
Laut Präsenzaufgaben gilt: $\log_{a}(x) = \frac{\log_{b}(x)}{\log_{b}(a)} <=> \log_{b}(a) = \frac{\log_{b}(x)}{\log_{a}(x)}$
also gilt: $\lim\limits_{n \to \infty}{\frac{log_{b}(n)}{log_{2}(n)}} = \lim\limits_{n \to \infty}{log_{b}(2)} =$ konstanter Wert für alle $b > 1$\\
\item
Behauptung:\\ wenn $f \in \Omega(g)$ ist, so ist $g \in \omega(f)$\\\\
Beweis:\\ Obige Behauptung ist äquivalent zu $\lim\limits_{n \to \infty}{\frac{f(n)}{g(n)}} < \infty $.
Dies gilt aber nur, wenn $g(n) > f(n)$ (da sonst $\lim\limits_{n \to \infty}{\frac{f(n)}{g(n)}}  = \infty$).
daraus folgt, dass der $\lim\limits_{n \to \infty}{\frac{g(n)}{f(n)}} = \infty$ ist und somit $g \in \omega(f)$ \\
\item 
Behauptung: \\ Für alle $c \in \mathbb{R+} $ und $f_{c}(n) := \sum\limits_{i=1}^{n}(c^i)$ gilt:\\ $f_{c}(n) \in O(n) \leftrightarrow c = 1$ \\\\
Beweis:\\ $f_{c}(n) \in O(n) \leftrightarrow \lim\limits_{n \to \infty}{\frac{f_{c}(n)}{n}} = $ konstanter Wert \\
$\rightarrow \lim\limits_{n \to \infty}{\frac{f_{c}(n)}{n}} = \lim\limits_{n \to \infty}\frac{{\sum\limits_{i=1}^{n}{(c^{i})}}}{n} = \lim\limits_{n \to \infty}{\frac{(c^{1} + c^{2} + \dots + c^{n})}{n}} $\\
$\rightarrow \sum\limits_{i=1}^{n}(c^i) > n$ für alle c ungleich 1 $\rightarrow  \lim\limits_{n \to \infty}\frac{{\sum\limits_{i=1}^{n}{(c^{i})}}}{n} = \infty $\\
$\rightarrow $Nur wenn $c = 1$ ist gilt: $\lim\limits_{n \to \infty}\frac{{\sum\limits_{i=1}^{n}{(c^{i})}}}{n} =$ konstanter Wert, denn dann gilt:$ \lim\limits_{n \to \infty}\frac{{\sum\limits_{i=1}^{n}{(c^{i})}}}{n} = \frac{1}{n}$ 
  \end{enumerate}
  \end{enumerate}
\end{enumerate}
\end{document}