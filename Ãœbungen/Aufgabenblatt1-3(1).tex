\documentclass{article}

\usepackage[utf8]{inputenc} \usepackage[ngerman]{babel}

\usepackage{amssymb} \usepackage{amsmath}

\usepackage{latexsym}

\title{Aufgabenblatt 1 - Aufgabe 3}

\author{}

\begin{document}

\maketitle

\begin{enumerate}
\item[(a)]
Zu beweisen: Es gilt für $n \geq 0$
\[
    \begin{pmatrix}
        F_{n} \\ F_{n+1}
    \end{pmatrix}
    =
    \begin{pmatrix}
        0 & 1 \\
        1 & 1
    \end{pmatrix}
    ^n \cdot
    \begin{pmatrix}
        F_0 \\ F_1
    \end{pmatrix}
\]

\underline{Induktionsanfang:} \quad $n = 0$
\begin{align*}
    \begin{pmatrix}
        0 & 1 \\
        1 & 1
    \end{pmatrix}
    ^0 \cdot
    \begin{pmatrix}
        F_0 \\ F_1
    \end{pmatrix}
    &=
    \begin{pmatrix}
        1 & 0 \\
        0 & 1
    \end{pmatrix}
    \cdot
    \begin{pmatrix}
        F_0 \\ F_1
    \end{pmatrix}
    \\
    &=
    \begin{pmatrix}
        F_0 \\ F_1
    \end{pmatrix}
    \hfill \square
\end{align*}

\underline{Induktionsbehauptung:} \quad Es gilt für $n$:
\begin{align*}   
    \begin{pmatrix}
        F_{n} \\ F_{n+1}
    \end{pmatrix}
    &=
    \begin{pmatrix}
        0 & 1 \\
        1 & 1
    \end{pmatrix}
    ^n \cdot
    \begin{pmatrix}
        F_0 \\ F_1
    \end{pmatrix}
\end{align*}

\underline{Induktionsschritt:} \quad $n \to n+1$
\begin{align*}   
    \begin{pmatrix}
        F_{n+1} \\ F_{n+2}
    \end{pmatrix}
    &=
    \begin{pmatrix}
        0 & 1 \\
        1 & 1
    \end{pmatrix}
    \cdot
    \begin{pmatrix}
        F_n \\ F_{n+1}
    \end{pmatrix}
    \\
    &=
    \begin{pmatrix}
        0 & 1 \\
        1 & 1
    \end{pmatrix}
    \cdot
    \left[
    \begin{pmatrix}
        0 & 1 \\
        1 & 1
    \end{pmatrix}
    ^n \cdot
    \begin{pmatrix}
        F_0 \\ F_1
    \end{pmatrix}
    \right]
    \\
    &=
    \left[
    \begin{pmatrix}
        0 & 1 \\
        1 & 1
    \end{pmatrix}
    \cdot
    \begin{pmatrix}
        0 & 1 \\
        1 & 1
    \end{pmatrix}
    ^n 
    \right]
    \cdot
    \begin{pmatrix}
        F_0 \\ F_1
    \end{pmatrix}
    \\
    &=
    \begin{pmatrix}
        0 & 1 \\
        1 & 1
    \end{pmatrix}
    ^{n+1} 
    \cdot
    \begin{pmatrix}
        F_0 \\ F_1
    \end{pmatrix}
\end{align*}

        
\item[(b)]
Man berechen $X^n$, indem man erst $X^2$ berechne, dies mit sich 
selbst multipliziere ($\widehat{=} X^4$), dies wiederum mit sich 
selbst multipliziere ($\widehat{=} X^8$)... 

Damit benötigt man zB für $X^{64}$ nur 6 Multiplikationen, 
dies entspricht einem Aufwand von $O(log_2(n))$.

\item[(c)]
Um $F_n$ mit dem Matrizen-Verfahren zu berechnen benötigt man 
$(n-1) \cdot (8 \text{ Multiplikationen und } 4 \text{ Additionen})$ 
um $\begin{pmatrix} 0 & 1 \\ 1 & 1 \end{pmatrix}^n$ zu berechnen, 
sowie $4$ Multiplikationen und $2$ Additionen, um das Ergebnis mit 
$\begin{pmatrix} F_0 \\ F_1 \end{pmatrix}$ zu verrechnen. 

Dies ergibt eine benötigte Zeit von 
\begin{align*}
&O((n-1) \cdot (8 \cdot 
64^{1.59} + 4 \cdot 64)+ 4 \cdot 64^{1.59} + 2 \cdot 64) \\ = 
&O((n-\frac{1}{2}) \cdot 8 \cdot 64^{1.59} + (n-\frac{1}{2})\cdot 4 \cdot 64) \\ 
= &O((8n-4)\cdot (64^{1.59 + 32}))
\end{align*}  
Dies ist immer noch linearer Aufwand, und damit asymptotisch 
echt schneller als $O(n^2)$
\end{enumerate}



\end{document}
