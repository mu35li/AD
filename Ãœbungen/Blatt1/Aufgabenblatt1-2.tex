\documentclass{article}

\usepackage[utf8]{inputenc} \usepackage[ngerman]{babel}

\usepackage{amssymb} \usepackage{amsmath}

\usepackage{latexsym}

\title{Aufgabenblatt 1 - Aufgabe 2}

\author{}

\begin{document}

\maketitle

\begin{enumerate}
\item[(a)]
Es gilt zu beweisen, dass $F_n \geq 2^{0.5 \cdot n}$ für alle $n \geq 6$ gilt.\\
\\

\underline{Induktionsanfang:} \quad $n = 6$ \\
\begin{equation*}
F_6 = 8 \geq 2^{0.5 \cdot 6} = 8
\end{equation*}

\underline{Induktionsannahme:} \quad $F_n \geq 2^{0.5 \cdot n}$ \\

\underline{Induktionsschritt:} \quad $n \to n+1$ \\
\begin{align*}
F_{n+1} = F_{n} + F_{n-1} &\geq 2^{0.5 \cdot n} \cdot 2^{0.5 \cdot (n-1)} \\
\Leftrightarrow F_{n} + F_{n-1} &\geq 2^{0.5 \cdot n} \cdot \left( 2^{0.5 \cdot
n} + 2^{0.5 \cdot -1} \right) \\
\Leftrightarrow F_{n} + F_{n-1} &\geq 2^{0.5 \cdot n} \cdot \left( 1 +
\frac{1}{\sqrt{2}} \right)
\end{align*}

Nun gilt es noch zu zeigen, dass $F_{n+1} \geq 2^{0.5 \cdot n+1}$ gilt, indem
wir zeigen, dass $2^{0.5 \cdot n} \cdot \left( 1 + \frac{1}{\sqrt{2}} \right)
\geq 2^{0.5 \cdot n+1}$ gilt. (Man kann unschwer sehen, dass aus letzterer
Ungleichung erstere folgt.)

\begin{align*}
2^{0.5 \cdot n} \cdot \left( 1 + \frac{1}{\sqrt{2}} \right) &\geq 2^{0.5 \cdot
n+1} \\
\Leftrightarrow 2^{0.5 \cdot n} \cdot \left( 1 + \frac{1}{\sqrt{2}} \right)
&\geq 2^{0.5 \cdot n} \cdot \sqrt{2} \\
\Leftrightarrow 1 + \frac{1}{\sqrt{2}} &\geq \sqrt{2} \\
\Leftrightarrow \frac{1 + \sqrt{2}}{\sqrt{2}} &\geq \sqrt{2} \\
\Leftrightarrow \frac{1 + \sqrt{2}}{\sqrt{2}} &\geq \frac{2}{\sqrt{2}} \\
\Leftrightarrow 1 + \sqrt{2} &\geq 2
\end{align*}

Da $\sqrt{2}$ größer als $1$ ist, ist $\sqrt{2}+1$ größer als $2$, woraus folgt,
dass die Behauptung stimmt. \quad $\square$

\item[(b)]
\begin{align*}
\text{Behauptung: } &B = F_n \le 2^{c \cdot n} \\
\text{Induktionsanfang: } & n = 0. F_n = 0 \le 2^{c \cdot 0} = 1 \\
\text{Induktionsannahme: } &\text{Gelte B für n. } \\
\text{Induktionsschritt: Zu zeigen: } F_{n+1} &\le 2^{c \cdot (n+1)} \\
\text{Wenn } F_n + F_{n-1} \le 2^{c \cdot n} + 2^{c \cdot (n-1)} &\le 2^{c \cdot (n+1)}, \\
\text{dann gilt auch } F_{n+1} = F_n + F_{n-1} &\le 2^{c \cdot (n+1)} \\
\text{Zu zeigen: }  2^{c \cdot n} + 2^{c \cdot (n-1)} &\le 2^{c \cdot (n+1)} \\
2^{c^n} + 2^{c \cdot n - c \cdot 1} &\le 2^{c^n} \cdot 2^c \\
2^{c^n} + 2^{c \cdot n} \cdot 2^{c^{-1}} &\le 2^{c^n} \cdot 2^c \\
1 + \frac{1}{2^c} &\le 2^c \\
\Rightarrow c &\ge \frac{log(1+\sqrt5)}{log(2)} -1 \\
\Rightarrow c &\ge \approx 0.7 \\
\text{Also gilt } &F_n \le 2^{c \cdot n} \text{ für } c \ge 0.7. \square

\end{align*}
\end{enumerate}
\end{document}
